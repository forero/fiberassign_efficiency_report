\documentclass{article}
\usepackage[utf8]{inputenc}

\title{Fiberassign efficiency October 2019}
\author{J. E. Forero-Romero}
\date{April 2020}

\begin{document}

\maketitle

\begin{abstract}
I use $400$ deg$^2$ of data from DR8 to estimate the fiberassign efficiency.
I compare three different versions of fiberassign and two different
strategies to run fiberassign during the survey. 
The simulation iteratively runs fiberassign on different layers,
builds the redshift catalog and updates the Merged Target List.
The efficiencies are estimated from the final redshift catalog after running 
on all the dark and gray layers.
The test was made on October 2019. This means that fiberassign, the focal
plane and the footprint correspond to versions that are currently deprecated 
(April 2020). 
The efficiency results are close to 98.5 \% for QSO, 94.5 \% for LRG and 69.9 \% for ELG. Different fiberassign codes and strategies change the efficiency by 1\%.
\end{abstract}
\section{Software Versions}

The test ran with the \verb'desi master' versions on October 2019. 
The scripts to prepare the data and derive the statistics are stored
under \verb"https://github.com/forero/multipass". 
The script \verb'run_multilayer.py' prepares the data and runs fiberasign while
\verb'stats_multilayer.py' derives the statistics. 

\section{Inputs}

\begin{itemize}
\item Sky positions: \\
\verb"/project/projectdirs/desi/target/catalogs/dr8/0.31.0/skies/skies-dr8-0.31.0.fits"
\item Targets:\\
\verb"/project/projectdirs/desi/target/catalogs/dr8/0.31.1/targets/main/resolve/"
\item Truth:\\
I assign a true spectral type to each target in the
following order:
\verb'MWS_ANY', \verb'BGS_ANY', \verb'STD_FAINT',
\verb'STD_BRIGHT',\verb'ELG', \verb'LRG', \verb'QSO'.
This means that if a target passed both cuts for, say, \verb'MWS_ANY' and
\verb'QSO' then it is assigned to be a \verb'QSO'.
Finally, I assign a number density of $50$ deg$^{-2}$
to Lya QSO targets.

\item Footprint:\\
I use the result of \verb'desimodel.io.load_tiles()'.
I use the \verb"DARK" and \verb"GRAY" tiles split into 
five different layers: \verb`gray`,
\verb'dark0', \verb'dark1', \verb'dark2' and \verb'dark3'.

\end{itemize}
For all the files we use a subset with coordinates 
$130<$ RA $<180$ and
$-10<$ dec $<40$. 
The statistics are derived within a region of coordinates 
$140<$ RA $<180$ and
$10<$ dec $<20$. (Yes, that's right. The upper limit in RA goes up the
to inital boundary in the input targets.)

\section{Simulation setup}

I use three different fiberassign executables.
\begin{itemize}
\item \verb"fiberassign_master": the version on \verb'master'.
\item \verb"fiberassign_legacy": the legacy version.
\item \verb'fiberassign_legacy_noimprove': the legacy version compiled
without the \verb'improve' step.
\end{itemize}

I group the tiles into four consecutive layers: \verb'gray', \verb'dark0',
\verb'dark1' and \verb'dark2+dark3'.
I use two different strategies to run fiberassign,  create the
redshift catalogs and update the MTL file. 
The first strategy (\textbf{Strategy A}) runs fiberassign for each
layer.
The second strategy (\textbf{Strategy B}) runs fiberassign for all
future layers.
In both cases the redshift catalog grows progressively with the tiles of each layer.

This means that for \textbf{Strategy B} I start running fiberassign for all tiles in layers \verb'gray+dark0+dark1+dark2+dark3' but use only the tiles in \verb'gray0' to build the first iteration of the redshift catalog. 
In contrast,  \textbf{Strategy A} starts running fiberassign for the \verb'gray' layer only.



\section{Results}

In the following tables I summarize the absolute numbers for 
each target and the number of observations it had. The last column
lists the efficiency computed as the number of targets with at least
one observation divided between the total number of available targets.
Targets that were out of reach are not taken into account.

\newpage

\subsection*{master strategy A}
\begin{center}
\begin{tabular}{l|ccccc|c}
 & N=0 & N=1 & N=2 & N=3 & N=4 & eff\\\hline
QSO & 1607 & 78521 & 11267 & 4640 & 12695 & 0.9852\\
LRG & 10220 & 133933 & 43485 & 0 & 0 &  0.9455\\
ELG & 303251& 706534 & 0& 0 & 0 & 0.6996
\end{tabular}
\end{center}

\subsection*{legacy strategy A}
\begin{center}
\begin{tabular}{l|ccccc|c}
 & N=0 & N=1 & N=2 & N=3 & N=4 & eff\\\hline
QSO & 1435& 78445& 11105& 4208& 13535 &  0.9868\\
LRG & 8665& 134223& 44751& 0& 0& 0.9538\\
ELG & 295499& 714284& 0& 0& 0 & 0.7073\\
\end{tabular}
\end{center}

\subsection*{legacy noimprove strategy A}
\begin{center}
\begin{tabular}{l|ccccc|c}
 & N=0 & N=1 & N=2 & N=3 & N=4 & eff\\\hline
QSO & 1433 &  78454 &  11113 &  4198 &  13530 &  0.9868\\
LRG & 8684 &  134206 &  44749 &  0 &  0 &  0.9537\\
ELG & 297186 &  712597 &  0 &  0 &  0 &  0.7056\\
\end{tabular}
\end{center}

\subsection*{master strategy B}
\begin{center}
\begin{tabular}{l|ccccc|c}
 & N=0 & N=1 & N=2 & N=3 & N=4 & eff\\\hline
QSO &  1607 &  78660 &  11122 &  4641 &  12700 &  0.9852\\
LRG &  10234 &  133939 &  43465 &  0 &  0 &  0.9454\\
ELG &  295059 &  714726 &  0 &  0 &  0 &  0.7078\\
\end{tabular}
\end{center}

\subsection*{legacy strategy B}
\begin{center}
\begin{tabular}{l|ccccc|c}
 & N=0 & N=1 & N=2 & N=3 & N=4 & eff\\\hline
QSO & 1445 &  78583 &  10953 &  4223 &  13524 &  0.9867\\
LRG & 8675 &  134208 &  44756 &  0 &  0 &  0.9537\\
ELG & 285927 &  723856 &  0 &  0 &  0 &  0.7168\\
\end{tabular}
\end{center}

\subsection*{legacy noimprove strategy B}
\begin{center}
\begin{tabular}{l|ccccc|c}
 & N=0 & N=1 & N=2 & N=3 & N=4 & eff\\\hline
QSO &  1431 &  78458 &  11106 &  4192 &  13541 &  0.9868\\
LRG &  8696 &  134180 &  44763 &  0 &  0 &  0.9536 \\
ELG &  297234 &  712549 &  0 &  0 &  0 &  0.7056\\
\end{tabular}
\end{center}


\end{document}
