\documentclass{article}
\usepackage[utf8]{inputenc}

\title{Fiberassign efficiency}
\author{J. E. Forero-Romero}
\date{April 2020}

\begin{document}

\maketitle

\section{Introduction}


\section{Software Versions}

The test ran with the \verb'desi master' versions on October 2019.

The scripts to prepare the data and derive the statistics are stored
under \verb"https://github.com/forero/multipass". The script
\verb'run_multilayer.py' prepares the data and runs fiberasign while
\verb'stats_multilayer.py' derives the statistics.

\section{Inputs}

\begin{itemize}
\item Sky positions: \\
\verb"/project/projectdirs/desi/target/catalogs/dr8/0.31.0/skies/skies-dr8-0.31.0.fits"
\item Targets:\\
\verb"/project/projectdirs/desi/target/catalogs/dr8/0.31.1/targets/main/resolve/"
\item Truth:\\
I assign a true spectral type to each target in the
following order:
\verb'MWS_ANY', \verb'BGS_ANY', \verb'STD_FAINT',
\verb'STD_BRIGHT',\verb'ELG', \verb'LRG', \verb'QSO'.
This means that if a target passed both cuts for, say, \verb'MWS_ANY' and
\verb'QSO' then it is assigned to be a \verb'QSO'.
Finally, I assign a number density of $50$ deg$^{-2}$
to Lya QSO targets.

\item Footprint:\\
I use the result of \verb'desimodel.io.load_tiles()'.
I use the \verb"DARK" and \verb"GRAY" tiles split into 
five different layers: \verb`gray`,
\verb'dark0', \verb'dark1', \verb'dark2' and \verb'dark3'.

\end{itemize}
For all the files we use a subset with coordinates 
$130<$ RA $<180$ and
$-10<$ dec $<40$. 

The statistics are derived within a region of coordinates 
$140<$ RA $<180$ and
$10<$ dec $<20$. (Yes, that's right. The upper limit in RA goes very
close to the bondary.)

\section{Simulation setup}

I use three different fiberassign executables.
\begin{itemize}
\item \verb"fiberassign": the version on \verb'master'.
\item \verb"fiberassign_legacy": the legacy version.
\item \verb'fiberassign_legacy_noimprove': the legacy version compiled
without the \verb'improve' step.
\end{itemize}

Furthermore I use two different survey strategies to run fiberassign, 
create the redshift catalogs and update the MTL file.

\section{Results}

In the following table I summarize the absolute numbers for 
each target and the number of observations it had. The last columns
lists the efficiency computed as the number of targets with at least
one observation divided between the total number of available targets.


\end{document}
